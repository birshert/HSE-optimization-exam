Задача условной оптимизации:
\[
    (P)
    \begin{cases}
        f(x) \to \min\limits_{x} \\
        g_{i}(x) \leq 0, \, i \in \{ 1, \dots , m \} \\
        h_{i}(x) = 0, \, j \in \{ 1, \dots , p \} \\
    \end{cases}
    \quad f, g_{i}, h_{j} \in C^{1}
\]
Функция Лагранжа:
\[
    L(x, \lambda, \mu) = f(x) + \sum\limits_{i=1}^{m} \lambda_{i} g_{i}(x) + \sum\limits_{j=1}^{p} \mu_{j} h_{j}(x)
\]
Двойственная функция Лагранжа:
\[
    q(\lambda, \mu) := \inf\limits_{x} L(x, \lambda, \mu)
\]
Свойства двойственной функции:
\begin{itemize}
    \item $q$ - вогнутая функция
    \item $q(\lambda, \mu) \leq f_{opt}$, для $\lambda \geq 0, \mu$
\end{itemize}
Двойственная задача оптимизации:
\[
    (D)
    \begin{cases}
        q(\lambda, \mu) \to \max\limits_{\lambda, \mu} \\
        \lambda \geq 0
    \end{cases}
\]
Двойственность:
\begin{itemize}
    \item \textbf{Слабая} двойственность: $q_{opt} \leq f_{opt}$
    \item \textbf{Сильная} двойственность: $q_{opt} = f_{opt}$
\end{itemize}
Если выполнено условие регулярности и $(P)$ является выпуклой задачей оптимизации, то выполняется сильная двойственность.
Пример построения двойственной задачи: \\
Исходная задача:
\[
    \begin{cases}
        c^{T} x \to \min\limits_{x \in \mathbb{R}^{n}} \\
        Ax \leq b, \quad A \in \mathbb{R}^{m \times n}, \Rk(A) = \min(m, n)
    \end{cases}
\]
Лагранжиан:
\[
    L(x, \lambda) = c^{T} x + \lambda^{T} (Ax - b) = (c + A^{T}\lambda)^{T} x - \lambda^{T} b
\]
Двойственная функция:
\[
    q(\lambda) = \inf\limits_{x \in \mathbb{R}^{n}} L(x, \lambda)
\]
Двойственная задача:
\begin{itemize}
    \item $c + A^{T}\lambda = 0$: \quad $\begin{cases}
                                             q(\lambda) = -\lambda^{T}b \to \max\limits_{y \in \mathbb{R}^{m}} \\ c + A^{T}\lambda = 0 \\ \lambda \geq 0
    \end{cases}$
    \item $c + A^{T}\lambda \neq 0$: $q(\lambda) = -\infty$
\end{itemize}