Для начала вспомним самое начало:
Для функции одной переменной $f: \mathbb{R} \to \mathbb{R}$ ее производная в точке $x$ обозначается $f'(x)$ и определяется из равенства:

$$f(x+h) = f(x) + f'(x)h + o(h)\ \text{для всех достаточно малых}\ h.$$

То есть при некоторой зафиксированной точке $x$ мы хотим приблизить изменение функции $f(x + h) - f(x)$ в окрестности нашего $x$ с помощью линейной функции по $h$, и $f'(x)h$ -- наилучший способ это сделать.
\bigskip

\textbf{Определение.} Пусть $x\in X$ -- внутренняя точка множества $X$ и пусть $L : U \to V$ -- линейный оператор. Будем говорить, что фукнция $f$ \textit{дифференцируема в точке x с производной L}, если для всех достаточно малых $h \in U$ справедливо следующее разложение:

$$f(x+h) = f(x) + L[h] + o(\|h\|).$$

Будем обозначать производную символом $df(x),$ также валидны символы $Df(x)[h]$ или $df(x)[h]$, где $h = \Delta x$ -- приращение.
\bigskip

\textbf{Определение.} Производная функции в точке $x$ --  это линейный оператор $df(x)$, который лучше всего аппроксимирует приращение функции:

$$f(x+h) - f(x) \approx Df(x)[h].$$

\textbf{Свойства производной}:

Пусть $U$ и $V$ -- векторные пространства, $X$ -- подмножество $U$, $x\in X$ -- внутренняя точка $X$. Справедливы следующие свойства:

\begin{itemize}
    \item (Линейность) Пусть $f : X \to V$ и $g : X \to V$ -- функции. Если $f$ и $g$ дифференцируемы в точке $x$, а $c_1,\ c_2\in\mathbb{R}$ -- числа, то функция $(c_1f + c_2g)$ также дифференцируема в точке $x$ и ее производная:

    $$d(c_1f + c_2g)(x) = c_1df(x) + c_2dg(x).$$

    \item (Правило произведения) Пусть $\alpha: X \to \mathbb{R}$ и $f : X \to V$ -- функции. Если $\alpha$ и $f$ дифференцируемы в точке $x$, то функция $\alpha f$ также дифференцируема в точке $x$ и

    $$D(\alpha f)(x)[h] = (D \alpha(x)[h])f(x) + \alpha(x)Df(x)[h].$$


    \item (Правило композиции) Пусть $Y$ -- подмножество $V$, $f : X \to Y$ -- функция. Также пусть $W$ -- векторное пространство, $g : Y \to W$ -- функция. Если $f$ дифференцируема в точке $x$, и $g$ дифференцируема в точке $f(x)$, то их композиция $(g \circ f) : X \to W$ (определяем как $g(f(x))$) дифференцируема в точке $x$, и

    $$D(g \circ f)(x) = Dg(f(x))[df(x)] $$
\end{itemize}
\bigskip

\textbf{Определение.} В случае $U = \mathbb{R}^n$ линейную функци. $Df(x)[h]$ можно представить с помощью скалярного произведения с некоторым вектором:

$$Df(x)[h] = \langle a_x, h\rangle,\ \ \ \text{где}\ a_x\in\mathbb{R}^n\text{ -- разный для каждого } x. $$

Вектор $a_x$ называется \textit{градиентом}  функции $f$ в точке $x$ и обозначается $\nabla f(x)$.

Отметим, что в стандартном базисе он представляется в привычном нам виде вектора (столбца) частных производных:

$$\nabla f(x) = \left(\dfrac{\partial f}{\partial x_1}(x),\ldots,\dfrac{\partial f}{\partial x_n}(x)\right)\in \mathbb{R}^n.$$
\bigskip

\textbf{Определение.} Пусть функция $f: X \to V$ дифференцируема в каждой точке $x\in X \subseteq U$. \\
Рассмотрим производную функции $f$ при фиксированном приращении $h_1\in U$ как функцию от $x$:

$$g(x) = Df(x)[h_1].$$

Если для функции $g$ в некоторой точке $x$ существует производная, то она называется \textit{второй производной} функции $f$ в точке $x$:

$$D^2 f(x) [h_1, h_2] := Dg(x)[h_2].$$
\bigskip

\textbf{Определение.} Вторую производную для функций $f :  \mathbb{R}^n\to\mathbb{R}$ также можно представить в виде матрицы:
$$D^2f(x)[h_1, h_2] = \langle H_xh_1, h_2\rangle,\ \ H_x\in\mathbb{R}^{n\times m}.$$

Матрица $H_x$ называется \textit{гессианом} функции $f$ в точке $x$и обозначается $\nabla^2f(x).$ В стандартном базисе это матрица частных производных:

$$\nabla^2f(x) = \left(\dfrac{\partial^2 f}{\partial x_i\partial x_j}(x)\right)_{i,j = 1}^n.$$

Для дважды непрерывно дифференцируемой функции гессиан симметричен.