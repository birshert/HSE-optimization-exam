\textbf{Сопряженная функция Фенхеля}. Пусть $f:~E \to \mathbb{R}$~-~функция, заданная на множестве $E$ в евклидовом пр-ве $V$. Сопряженной функцией для функции $f$ наз-ся функция $f^{*}:~E_{*} \to \mathbb{R}$, определенная как
$$
f^{*}(s):=\underset{x \in E}{sup}\{<s,x>-f(x)\},
$$
где $E_{*}:=\{s \in V:~sup_{x \in E}\{<s,x>-f(x) < +\infty\}\}$. Её и называют сопряж. функцией Фенхеля, а преобразование $f \to f^{*}$ зовут преобр. Фенхеля/Лежандра.\\
Сопряж ф-ия также зависит от скалярного произведения в пр-ве $V$, про него стоит дополнительно пояснять. Везде дальше в курсе по дефолту - стандартное.\\

\textbf{НЕравенство Фенхеля-Юнга}. Пусть $f:~E \to \mathbb{R}$~-~функция, заданная на множестве $E$ в евклидовом пр-ве, и пусть $f^{*}:~E_{*} \to \mathbb{R}$~-~сопряженная функция. Тогда для всех $x \in E$ и всех $s \in E_{*}$~выполнено:
$$
<s,x> \leq g(s) + f(x).
$$
Несмотря на свою простоту, это утв. иметт многочисленные приложения. Нер-во Фенхеля-Юнга дает систематичный способ построения оценок вида
$$
<s,x> \leq f^{*}(s)+f(x),~~~~(1)
$$
для всех $s$ и $x$, где $f$ и $g$~-~некоторые функции. Необходимость сопрженной функции можно связать с таким вопросом: какие пары функций $f,g$ нужно брать в нер-ве $(1)$, чтобы оно было "максимально точным"?. Если зафиксировать конкретным значением функцию $f$, то другая функция $g$, дающая максимальную "точную" оценку в нер-ве $(1)$ уже определяется однозначно - в точности сопряженная функция $f^{*}$.\\
Пример с поиском сопряженной функции: пусть $V$~-~евклидово пр-во, и пусть $f:~V \to \mathbb{R}$~-~функция $f(x):=<a,x>$, где $a \in V$. Поскольку
$$
\underset{x \in V}{sup}\{<s,x>-<a,x>\}=\underset{x \in V}{sup}<s-a,x>=\begin{cases}
0,\text{~если~}s=a,\\
+\infty,\text{~иначе~}
\end{cases}
$$
для всех $s \in V$, то сопряженная функция $f^{*}$~-~это индикаторная функция $\delta_{\{a\}}$ одноэлементного мн-ва $\{a\}$. И пример наоборот:\\

Пусть $V$~-~евклидово пр-во, $a \in V$, и пусть $\delta{\{a\}}:~\{a\} \to \mathbb{R}$~-~индикаторная функция одноэлементного мн-ва $\{a\}$. Поскольку для $s \in V$ выполнено
$$
\underset{x \in \{a\}}{sup}\{<s,x>-\delta_{\{a\}}(x)\}=<s,a>,
$$
то сопряженная функция $\delta^{*}_{\{a\}}$ равна линейной форме $x \to <a,x>$, рассмотренной в предыдущем примере.\\

В след теормине будут несколько примеров, которые показывают, что преобразование Фенхеля это инволюция, т.е. сопряженная функция $f^{**}$ к функции $f^{*}$~-~это исходная функция $f$. На основе этой идеи существует теорема Фенхеля-Моро.\\

\textbf{Теоремка Фенхеля-Моро}. Пусть $f$~-~функция, заданная на непустом множестве в евклидовом пространстве. Тогда $f=f^{**}$, если и только если $f$ является выпуклой и замкнутой.\\

В одну сторону теоремка очевидна: если $f=f^{**}$, то $f$ обязана быть выпуклой и замкнутой, т.к. сопряженная сопряж. функция к любой функции (в том числе к $f^{*}$) является выпуклой и замкнутой вне зависимости от того, обладала ли исходная функция этими свойствами или нет. Нетривиальный частью теоремы Фенхеля-Моро является обратное утв., что замкнутость $f$ является не только необходимым условием, но и достаточным. (хз как доказывать в эту сторону, наверно уточним у Кропотова, надо ли это делать).\\

Список кучи полезных свойств с сопряженными функциями. Пусть $f:~E \to \mathbb{R}$ и $g:~G \to \mathbb{R}$~-~функции, заданные на множествах $E$ и $G$ в евклидовых пр-вах $V$ и $W$ соответственно, и пусть $f^{*}:E_{*} \to \mathbb{R}$ и $g^{*}:~G_{*} \to \mathbb{R}$~-~соответствующие сопряженне функции.\\
\begin{enumerate}
    \item (Умножение на полож. скаляр) Если $c > 0$, то $(cf)^*:~cE_{*} \to \mathbb{R}$~-~это функция $(cf)^*(s)=cf^*(s/c)$ для всех $s \in E_{*}$
    \item (Сдвиг аргумента) Пусть $a \in V$, и пусть $h:~E - a \to \mathbb{R}~-~функция h(x):=f(x+a)$. Тогда $h^*:~E_* \to \mathbb{R}$~-~это функция $h^*(s)[f^*(s)-<s,a>$ для всех $s \in E_*$.
    \item (Прибавление аффинной функции) Пусть $a \in V,~b \in mathbb{R}$ и пусть $h:~E \to \mathbb{R}$~-~функция $h(x):=f(x)+<a,x>+b$. Тогда $h^*:E_*+a \to \mathbb{R}$~-~это функция $h^*(s)=f^*(s-a)-b$ ждя всеъ $s \in E_* + a$.
    \item (Композиция с обратимым лин. преобразованием) Пусть $A:~V \to V$~-~обратимое линейное преобразование. Тогда $(f \circ A)^*:~A^*(E_*) \to \mathbb{R}$~-~это функция $(f \circ A)*(s)=f^*(A^{-*}s)$ для всех $s \in A^*(E_*)$.
    \item (Сепарабельная сумма). Пусть $\phi:E \times G \to \mathbb{R}$~-~функция $\phi(x,y):=f(x)+g(y)$. Тогда $\phi^*:E_* \times G_* \to \mathbb{R}$~-~это функция $\phi_*(s,u)=f^*(s)+g^*(u)$ для всеъ $s \in E_*$ и всех $u \in G_*$.
\end{enumerate}

На применение этих исчислений могут спокойно дать задачу какую-нибудь.\\

Перед выводом связи меж сопряж ф-ей и субдифф-ом стоит рассмотреть след пример. Пусть $V$~-~конечномерное евклидово пр-во, $x_0 \in V$. Пусть $||~||$~-~произвольная норма в $V$ (необязательно индуцированная скалярным произведением), и $||~||_{*}$~-~соотв. сопряженная норма. Тогда:
$$
\partial||~||(x_0) = \{s \in V:||s||_{*} \leq 1; <s,x_0>=||x_0||\}=\begin{cases}
\overline{B_{||~||}},(0,1),\text{если}~x_0=0\\
{s \in V: ||s||_{*}=1;~<s,x_0>=||x_0||}, \text{иначе}
\end{cases}~~~~(1)
$$
Где $\overline{B_{||~||}},(0,1):=\{s \in V:||s||_{*} \leq 1\}$~-~замкнутый единичный шар с центром в нуле относительно сопряженной нормы. Другими словами, вектор $s \in V, ||s||_{*}=1$, является субградиентом нормы $||~||$ в точке $x_0 \neq 0$, если и только если нер-во Гельдера (след пункт теормина)~$<s,x_0> \leq ||x_0||$~переходит в равенство.\\

\textit{Док-во}. Пусть $s \in V$. По опр субградиента: $s \in \partial ||~||(x_0)$, если и только если $$
<s,x>-||x|| \leq <s,x_0>-||x_0||
$$
,для всех $x \in V$, или, квивалентно
$$
\underset{x \in V}{sup}\{<s,x>-||x||\} \leq <s,x_0>-||x_0||.~~~(*)
$$
По определению супремума, последнее равносильно(тут даже автор конспекта чет сомневается)
$$
\underset{x \in V}{sup}\{<s,x>-||x||\}=<s,x_0>-||x_0||
$$
Заметим, что выражение в левой части есть супремум из опредеелния сопряженной функции Фенхеля для нормы, который, как мы уже знаем, равен:
$$
\underset{x \in V}{sup}\{<s,x>-||x||\}=\begin{cases}
0, \text{если} ||s||_{*} \leq 1,\\
+\infty, \text{иначе}
\end{cases}
$$
Таким образом, $(*)$~эквивалентно
$$
||s||_{*} \leq 1~~\text{и}~~<s,x_0>=||x_0||.
$$
В итоге
$$
\partial||~||(x_0)=\{s \in V:||s||_{*}\leq 1; <s,x_0>=||x_0||\}.
$$
Остается заметить, что для $x_0 \neq 0$~нер-во $||s||_{*} \leq 1$~обязано переходить в равенство, поскольку при $||s||_{*} < 1$, из нер-ва Гельдера следует $<s,x_0> \leq ||s||_{*}||x_0|| < ||x_0||$. (одним словом - ужас).\\
На этом примере с сопряженной нормой можно понять, что аналогичным образом для произвольной функции $f$ (т.е. не только для нормы) ее субдиф можно описать в терминах двойственного объекта - сопряженной функции Фенхеля.\\

Тогда в сухом остатке эта формулировка звучит так: Пусть $f: E \to \mathbb{R}$~-~функция, определенная на мн-ве $E$ в евклидовом пр-ве. Пусть $x_0 \in E$, и пусть $f^*:E \to \mathbb{R}$~-~сопряженная функция. Тогда можно показать, что
$$
\partial f(x_0)=\{s \in E_{*}:<s,x_0>=f^{*}(s)+f(x_0)\}~~~(2)
$$
Словами: вектор $s \in E_{*}$~является субградиентом функции $f$ в точке $x_0$, если и только если неравенство Фенхеля-Юнга $<s,x_0> \leq f^{*}(s)+f(x_0)$~переходит в равенство.\\
В случае $f=||~||$, получаем $f^{*}=\delta_{\overline{B_{||~||}},(0,1)}$, т.е. сопряженная функция равна индикаторной функции шара $\overline{B_{||~||}},(0,1)$ и формула выше~$(2)$~переходит в формулу из примера~$(1)$~.\\

Следствие из этих рассуждений - критерии равенства в нер-ве Фенхеля-Юнга. Пусть $f:~R \to \mathbb{R}$~-~выпуклая замкнутая функция, $f^{*}:E_{*} \to \mathbb{R}$~-~сопряженная функция, и пусть $x \in E,~s \in E_{*}$. Тогда след. утверждения эквивалентны:\\
1. $<s,x>~=~f^{*}(s)+f(x)$\\
2. $s \in \partial f(x)$.\\
3. $x \in \partial f^{*}(s)$.\\
\textit{Док-во}. Согласно выводу по связи сопряж функции и субдиф-ов, условие $<s,x>=f^{*}(s)+f(x)$~равносильно вложению $s \in \partial f(x)$. С другой стороны, поскольку $f$ выпуклая и замкнутая, то, по теореме Фенхеля-Моро, $f^{**}=f$. Применяя вывод по связи выше к функции $f^{*}$, получаем, что равенство $<s,x>=f^{*}(s)+f(x)$~эквивалентно вложению $x \in \partial f^{*}(s)$.

Открытый вопрос: надо ли в рамках этого пункта что-то про двойственность фенхеля говорить. Ответ узнаем завтра на консе.