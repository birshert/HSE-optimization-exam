Субдифференциал функции: 
$$\partial F(x)=\{z|F(y) \geq F(x) + z^T(y-x)~\forall y\}$$
Фактически это множество касательных к функции в точке $x$. Если функция гладкая и дифференцируемая в каждой точке, то в этом множестве будет всего одна касательная - градиент к функции в каждой её точке. Т.е.
$$
\text{Если}~F(x) \in C^1 \to \partial F(x)=\{\nabla F(x)\}
$$
Если же функция просто гладкая, но производная есть не в каждой её точке - тогда и будет получено данное мн-во.\\
Пример:\\
$$
F(x)=|x| \to \partial F(x)\begin{cases}
x > 0,~\{1\}\\
x < 0,~\{-1\}\\
x = 0.~[-1,1]
\end{cases}
$$
Здесь в качестве $z$ получены коэф-ты к касательным в каждой точке функции. Последний случай также очень легко проверяется:
$$
F(y) \geq F(0) + z^T(y-0)~\forall y \to |y| \geq z*y \to \begin{cases}
1)~y>0 \to z \leq 1\\
2)~y < 0 \to z \geq -1
\end{cases}
$$
Также субдифференциал это выпуклое мн-во, проверяется легко через его определение:
$$
z_1,z_2 \in \partial F(x) \to \begin{cases}
F(y) \geq F(x) + z_1^T(y-x)\\
F(y) \geq F(x) + z_2^T(y-x)
\end{cases}~\forall y \to
$$
$$
\alpha F(y) + (1-\alpha) F(y) \geq \alpha(F(x) + z_1^T(y-x)) + (1-\alpha)(F(x) + z_2^T(y-x)) \to
$$
$$
F(y) \geq F(x) + (\alpha z_1 + (1-\alpha)z_2)^T(y-x) \to \alpha z_1 + (1-\alpha)z_2
$$
Субградиент - эл-т из мн-ва субдифференциала: $z \in \partial F(x)$.\\