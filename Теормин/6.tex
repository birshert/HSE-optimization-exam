Субдифференциал функции:
$$\partial F(x)=\{z|F(y) \geq F(x) + z^T(y-x)~\forall y\}$$
Фактически это множество касательных к функции в точке $x$. Если функция гладкая и дифференцируемая в каждой точке, то в этом множестве будет всего одна касательная - градиент к функции в каждой её точке. Т.е.
$$
\text{Если}~F(x) \in C^1 \to \partial F(x)=\{\nabla F(x)\}
$$
Если же функция просто гладкая, но производная есть не в каждой её точке - тогда и будет получено данное мн-во.\\
Пример:\\
$$
F(x)=|x| \to \partial F(x)\begin{cases}
                          x > 0,~\{1\}\\
                          x < 0,~\{-1\}\\
                          x = 0.~[-1,1]
\end{cases}
$$
Здесь в качестве $z$ получены коэф-ты к касательным в каждой точке функции. Последний случай также очень легко проверяется:
$$
F(y) \geq F(0) + z^T(y-0)~\forall y \to |y| \geq z*y \to \begin{cases}
                                                         1)~y>0 \to z \leq 1\\
                                                         2)~y < 0 \to z \geq -1
\end{cases}
$$
Также субдифференциал это выпуклое мн-во, проверяется легко через его определение:
$$
z_1,z_2 \in \partial F(x) \to \begin{cases}
                              F(y) \geq F(x) + z_1^T(y-x)\\
                              F(y) \geq F(x) + z_2^T(y-x)
\end{cases}~\forall y \to
$$
$$
\alpha F(y) + (1-\alpha) F(y) \geq \alpha(F(x) + z_1^T(y-x)) + (1-\alpha)(F(x) + z_2^T(y-x)) \to
$$
$$
F(y) \geq F(x) + (\alpha z_1 + (1-\alpha)z_2)^T(y-x) \to \alpha z_1 + (1-\alpha)z_2
$$
Субградиент - эл-т из мн-ва субдифференциала: $z \in \partial F(x)$.\\
Пусть $f~:~E \to \mathbb{R}$~-~функция, определенная на множестве $E$ в евклидовом пр-ве $V$, и пусть $x_0 \in E$. Вектор $s \in V$ называется {\bf субградиентом} функции $f$ в точке $x_0$, если:
$$
f(x) \geq f(x_0) + <s, x-x_0>~\forall x \in E
$$
Это определение субградиента функции в точке $x_0$ в отрыве от определения субдифференциала. Из этих определений следует, что субград/субдиф зависят от введенного в пр-ве $V$ скалярного произведения. Поэтому если в рамках задачи/теории не оговорено иное, то под скалярным произведением в рассматриваемом $V$ мы берем стандартное.\\
При использовании субграда можно перейти к терминологии, что это часть двойственного пр-ва ($s \in V^{*}$), а $<s,x>$~-~вычисление лин функционала $s$ на аргументе $x$ (это позволят рассматривать такие $V$, где скалярное произведение явно ввести невозможно), но это все сложно и для упрощения теории далее не рассматривается.\\
Также важно упомянуть, что из субдифференцируемости функции (т.е. в каждой её точке есть субдифференциал как непустое мн-во) следует выпуклость этой функции (проверяется через опр выпуклости и субдиф-ла в точках $x,~y$). Но обратное утверждение неверно. Пример:
$$
\text{Пусть}~f:~[0,+\infty) \to R;~f(x)=-\sqrt{x}. \text{~Тогда~} \partial f(0)=\emptyset
$$
Тут также распишете субдиф в нуле по определению и получите, что $-\sqrt{x} \geq z^T(x - 0) \to zx \leq -\sqrt{x} \to z \leq -\frac{1}{\sqrt{x}} \to z \leq -\infty$, что невозможно. Поэтому в обратную сторону эта тема не работает.\\

Не всякая выпуклая функция является субдифференцируемой.Но чаще всего, отсутствие субградиентов у выпуклой функции — это довольно экзотическая ситуация, которая происходит только в граничных точках области определения. Один из основных результатов теории субдифференциалов состоит в том, что в любой внутренней точке области определения выпуклая функция гарантированно имеет субградиенты, причем субдифференциал является не просто выпуклым и замкнутым множеством, но также ограниченным (и, тем самым, компактным, согласно теореме Гейне–Бореля).\\

Из этого дальше вытекает такая теорема. Пусть $f:~E \to \mathbb{R}$~-~выпуклая функция, определенная на множестве $E$ в конечно мерно евклидовом пр-ве $V$, и пусть $x_0 \in int(E)$. Тогда множество $\partial f(x_0)$~-~не пусто и является выпуклым и компактом.\\
Дальше есть что-то про связь субдиффов и сопряженных норм, но это просят в рамках друго пункта теормина, поэтому дальше к субдифф исчислениям.\\

Умножение на полож. скаляр. Пусть $f:~E \to \mathbb{R}$~-~функция, определенная на мн-ве $E$ в евклидовом пр-ве, $c > 0$, и $x_0  \in E$. Покажем, что
$$
\partial (cf)(x_0)=c \partial f(x_0).
$$
Это вроде легко показывается через определение субдифференциала и рассмотрение субградиента из него в общем виде. (поправьте, если это не так :) )\\

Одним из самых первых результатов, появившихся в теории субдифференциального исчисления, является правило суммы, которое в лит-ре известно под названием теоремы Моро-Рокафеллара.\\

\textbf{Теоремка Моро-Рокафеллара}. Пусть $f:~R \to \mathbb{R}$ и $g:~G \to \mathbb{R}$~-~функции, определенные на мн-вах $E$ и $G$ в одном и том же евклидовом пр-ве, и пусть $x_0 \in E \cap G$. Тогда
$$
\partial (f + g)(x_0) \supseteq \partial f(x_0)+\partial g(x_0)
$$
Ксли дополнительно функции $f$ и $g$ выпуклые, и при этом $E \cap int (G) \neq \emptyset$, то
$$
\partial (f+g)(x_0)=\partial f(x_0)+\partial g(x_0).
$$
Для док-ва вложения~(показано выше)~используют опр. субдифференциала через нер-ва. Для док-ва обратного вложения подключают теорему об отделимости выпуклых множеств.\\

Пример про важность $E \cap int(G) \neq \emptyset$:\\

\textbf{Упражнение}. Пусть $f:~[0,+\infty) \to \mathbb{R}$~-~функция $f(x):=-\sqrt{x}$, и пусть $g:~(-\infty,0] \to \mathbb{R}$~-~функция $g(x):=-\sqrt{-x}$. Покажите, что $\partial(f+g)(0)=\mathbb{R}$, в то время как $\partial f(0)=\partial g(0)=\emptyset$.\\

Используя индукцию и свойство о том, что внутренность конеч. пересечения множеств равна пересечению внутренностей, получаем обощение теоремы Моро-Рокафеллара для конечных сумм:\\

\textbf{Следствие, теорема Моро-Рокафеллара II}. Пусть $m \geq 2$~-~целое, $f_1:~E_1 \to \mathbb{R},...,f_m:~E_m \to \mathbb{R}$~-~функции, определенные на множествах $E_1,..,E_m$~в одном и том же евклидовом пр-ве, и пусть $x_0 \cap^m_{i=1}~E_i$. Тогда
$$
\partial \left( \sum^{m}_{i=1}\right)(x_0) \supseteq \sum^{m}_{i=1}\partial f_i(x_0).
$$
Если дополнительно функции $f_i$~-~выпуклые, и при этом $\cap^{m-1}_{i=1} int(E_i) \cap E_m \neq \emptyset$, то
$$
\partial \left(\sum^{m}_{i=1} f_i\right)(x_0)=\sum^{m}_{i=1}\partial f_i(x_0).
$$
Композиция с афинным преобразованием:\\
Пусть $V$ и $W$~-~евклидовы пр-ва,$T:~V \to W$~-~аффинное преобр. $T(x):=Ax+b$, где $A:~V \to W$~-~линейное преобр.,$b \in W$. Пусть $g:~G \to \mathbb{R}$~-~функция, определенная на множестве $G$ в пр-ве $W$ и $x_0 \in T^{-1}(G)$. Тогда
$$
\partial (g \circ T)(x_0) \supseteq A^{*}\partial g(T(x_0)),
$$
где $A^{*}:~W \to V$~-~сопряженный оператор для $A$. Если дополнительно функция $g$ выпуклая, и при этом $T(V) \cap int(G)\neq \emptyset$, то
$$
\partial (g \circ T)(x_0)=A^{*}\partial g(T(x_0)).
$$
\textbf{Пример}. Пусть $f:~\mathbb{R}^n \to \mathbb{R}$~-~функция $f(x):=|<a,x>|$, где $a \in \mathbb{R}^n$. Пусть $A:~\mathbb{R}^n \to R$~-~линейное преобр. $A(x):=<a,x>$. В этом случае $A^{*}t=ta$~для всех $t \in \mathbb{R}$(опять тут от автора "почему?"). Поскольку $A(\mathbb{R}^n) \cap int(\mathbb{R})\neq \emptyset$~(например, это мн-во содержит ноль), то
$$
\partial f(x)=\partial ||(<a,x>)a=\begin{cases}
[-1,1]
                                  a,\text{~если~}<a,x>=0,\\
                                  {sign(<a,x>)a},~\text{иначе}
\end{cases} = \begin{cases}
[-a,a]
              ,\text{~если~}<a,x>=0,\\
              {sign(<a,x>)a},~\text{иначе}
\end{cases}
$$
(Второе равенство вытекает из формулы для субдиф-ла модуля, пример выше)\\
функция $g$ это просто модуль, а скалярное произведение внутри $g$ и есть афинное преобр, с которым функция комбинируется. Выпуклость обеих функций очевидна, плюс пересечение результат лин опреатора и выпуклого мн-ва от $\mathbb{R}$ также понятны. Поэтому далее субдиф для функции $g$ вычисляется как от композиции функции и аффин. преобр. или как сопряж лин. оператор на субдиф от функции $g(T(x_0))$.\\

Для поиска субдифференциала от функций максимума используют следюущих два крутых правила:\\

\textbf{Теорема, максимум конечного числа функций}. Пусть $f_1:~E_1 \to \mathbb{R},...,f_m:~E_m \to \mathbb{R}$~-~функции, опр. на мн-вах $E_1,...,E_m$~в одном и том же конечномерном евклидовом пр-ве, пусть $f:~\cap^m_{i=1}~E_i \to \mathbb{R}$~-~функция
$$
f(x):=max\{f_1(x),...,f_m(x)\},
$$
и пусть $x_0 \in \cap^m_{i=1}~E_i$. Тогда
$$
\partial f(x_0) \supseteq \overline{Conv}(\cup_{i \in I(x_0)}\partial f_i(x_0)),
$$
где $I(x_0)$~-~мн-во индексов, на которых достигается максимум в точке $x_0$ (гарантированно непустое):
$$
I(x_0):=\{1 \leq i \leq m:~f_i(x_0)=f(x_0)\}.
$$
Если дополнительно функции $f_1,..,f_m$~-~выпуклые, и при этом $x_0 \in int(\cap^m_{i=1}~E_i)$~(или, эквивалентно, $x_0 \in \cap^m_{i=1}~int(E_i)$), то
$$
\partial f(x_0)=Conv(\cup){i \in I(x_0)}\partial f_i(x_0)).
$$
Теорема местами носит название ДУбовицкого-Милютина.\\

Пример. Пусть $f:~\mathbb{R} \to \mathbb{R}$~-~функция $f(x):=|x|=max\{-x,x\}$, и пусть $x_0 \in \mathbb{R}$. Применяя теорему выше и идею о том, что субдифференцируемая функция в каждой точке - выпуклая, получим:
$$
\partial f(x_0)=\begin{cases}
                Conv(\{-1,1\}), \text{~если~}x_0=0,\\
                Conv({sign(x_0)}),\text{~если~}x_0 \neq 0
\end{cases} = \begin{cases}
[-1,1]
              ,\text{~если~}x_0=0,\\
              {sign(x_0)},\text{~если~}x_0 \neq 0
\end{cases}
$$
Что совпадает с результатом посчитанного субдиф-ла для модуля ранее.\\

Для того, чтобы распространить предыдущее правило на беск семейство функций, необходимы дополнительные топологические условия: компактность и полунепрерывность.\\

\textbf{Теорема Данскина}. Пусть $V$~-~конечномерное евклидово пространство, $I$~-~произвольное множество (не обязательно конечное и не обязательно счетное), и пусть для каждого $i \in I$~задано множество $E_i$ в $V$ и функция $f_i:~E_i \to \mathbb{R}$. Пусть $f:~E \to \mathbb{R}$~-~функция
$$
f(x):=\underset{i \in I}{sup}f_i(x),
$$
определенная на мн-ве $E := \{x \in \cap_{i \in I}E_i:~sup_{i \in I}f_i(x)<+\infty\}$, и пусть $x_0 \in E$. Тогда
$$
\partial f(x_0) \supseteq \overline{Conv}(\cup_{i \in I(x_0)}\partial f_i(x_0)),
$$
где $I(x_0)$~-~множество индексов, на которых достигается супремум в точке $x_0$ (возможно, пустое):
$$
I(x_0):=\{i \in I:~f_i(x)=f(x)\}.
$$
Если дополнительно функция $f_i$ выпукла для каждого $i \in I$, мн-во $I$ является компактом (в некотором топологическом пространстве), и при этом для каждого $x \in \cap_{i \in I}E_i$~функция $i \to f_i(x)$ является полунепрерывной сверху на $I$, то $E=\cap_{i \in I}E_i$, мн-во $I(x_0)$ не пусто, и для $x_0 \in int(E)$ имеет место
$$
\partial f(x_0)=Conv(\cup_{i \in I(x_0)}\partial f_i(x_0)).
$$
В этой теореме помимо выпуклости требуются компактность индексного мн-ва $I$, а также полунепрерывность сверху функции $i \to f_i(x)$ на $I$ для каждого $x \in E$. Данные ограничения выглядят достаточно логично, если мы хотим, чтобы в каждой точке $x \in E$ супремум по $i \in I$ достигался хотя бы для одного из индексов, это будет следовать из того факта, что полунепрерывная сверху функция, опр на компакте, будет достигать своего наибольшего значения на нём. Особый интерес может представлять тот факт, что никаких других дополнительных ограничений при этом накладывать не надо: имещихся условий будет достаточно для вычисления дифференциала.\\

\textbf{Упражнение}. Пусть $I=\mathbb{R_{++}}$ и пусть $f_i(x)=-i \log x$. Тут предлагают проверить, выполняются ли условия теоремы Данскина. Как выглядит ф-ия $f(x)=sup_{i > 0}f_i(x)$? И стоит посмотреть $\partial f(1)$.\\

\textbf{Пример}. Снова попытаемся вывести субдифференциал для нормы, но уже через теорему Донскина.\\
Пусть $V$~-~конечномерное евклидово пространство, $||~||$~-~произвольная норма в V, и пусть $x_0 \in V$. Используя сопряж норму $||~||_{*}$ и тот факт, что сопряжение является инволюцией (боже, это просто досвидание!!!), можно записать след. вариационное представление:
$$
||x|| = \underset{||s||_{*}=1}{max}<s,x>
$$
справедливое для любого $x \in V$. В этом случае индексное мн-во $I := \{s \in V:~||s||_{*}=1\}$~представляет собой единичную сферу и является компактом (в пр-ве $V$). Поскольку $x_0 \in int(V)$, и для каждого $x \in V$ функция $s \to <s,x>$ является линейной (а, значит, непрерывной, и тем более, полунепрерывной сверху), то, согласное теореме Донскина, получаем
$$
I(x_0)=\{s \in V:~||s||_{*}=1;~<s,x_0>=||x_0||\},
$$
и
$$
\partial||~||(x_0)=Conv(\{s \in V:~||s||_{*}=1; <s,x_0>=||x_0||\}).
$$
Здесь была использована формула $\partial<s,>(x_0)={s}$.\\
Если $x_0=0$, то условие $<s,x_0>=||x_0||$ выполняется бессодержательно, и
$$
\partial||~||(0)=Conv(\{s \in V:~||s||_{*}=1\})=\overline{B}_{||~||_{*}}(0,1),
$$
(получили снова какой-то индикаторный оператор для шара). Если же $x_0 \neq 0$, то множество
$$
\{s \in V:~||s||_{*}=1:~<s,x_0>=||x_0||\}
$$
является выпуклым (снова почему? - думаю тут можно по определению выпуклости показать). Тогда операцию взятия выпуклой оболочки можно убрать, и в тоге получаем:
$$
\partial||~||(x_0)=\begin{cases}
                   \overline{B}_{||~||_{*}}(0,1),\text{~если~}x_0=0,\\
                   \{s \in V:~||s||_{*}=1:~<s,x_0=||x_0||>\},\text{~иначе~}.
\end{cases}
$$
В общем из всего, что тут есть - надо знать опредеелния субградиента/субдифференциала и всех теоремок из их исчисления. Примеры даны для понимания применения этих теоремок, может задачу вам на что нибудь такое дадут (правда тогда это вообще поминки будут).