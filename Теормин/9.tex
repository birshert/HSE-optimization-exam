Пусть $f: E \to \mathbb{R}$  - функция, заданная на множестве $E$ в евклидовом пространстве $V$. Определим проксимальный оператор функции $f$ как отображение $\text{Prox}_f : V \to E$, заданное следующим образом: $$ \text{Prox}_f(x) := \arg\min\limits_{y \in E} \left\{f(y) + \frac{1}{2} ||y-x||^2 \right \}$$ Если функция $f$ выпуклая и замкнутая, то в каждой точке множество значений $\text{Prox}_f$ состоит из одного элемента как множество минимумов сильно выпуклой и замкнутой функции. 

Рассмотрим функцию $g: E \to \mathbb{R}$, определенную как $$g(x) = \delta_C (x)$$ т.е. $g$ - индикаторная функция непустого множества $C \subseteq E$. Запишем проксимальное отображение для данной функции: $$\text{Prox}_g(x) = \arg\min\limits_{y \in E} \left\{ \delta_C(y) + \frac{1}{2}||y-x||^2 \right\} = \arg\min\limits_{y \in C} ||y-x||^2 = P_C(x)$$   Таким образом, проксимальный оператор для индикаторной функции данного множества есть оператор ортогональной проекции на данное множество. Если $C$ - замкнутое и выпуклое (а также непустое), то проецирующее отображение существует и единственно в каждой точке.
