\textbf{$LU$-разложение матрицы $A$}:
$$ A = LU$$где $L$ - нижнетреугольная матрица, $U$ - верхнетреугольная матрица. Собственно, разложение определено только для обратимых матриц с невырожденными главными минорами. В общем случае разложение определено неоднозначно - для однозначности можем потребовать унитреугольность матрицы $L$ или $U$. Используется для нахождения определителя, решения СЛАУ и обращения матриц. Сложность $\approx \frac{2}{3} n^3$.

\textbf{Разложение Холецкого}:
$$ A = LL^T$$ $A$ - симметричная положительно определенная матрица, $L$ - нижнетреугольная матрица (можно также записать как $U^T U$ через верхнетреугольную матрицу). Данное разложение единственно для любой удовлетворяющей условиям матрицы. Применяется для решения СЛАУ - в отличие от $LU$, здесь требуется примерно в 2 раза меньше итераций ($\frac{1}{3}n^3$) , при этом имеет место большая численная устойчивость. Часто применяется для решения задачи наименьших квадратов. Также используется в квази-ньютоновских методах.

Разложение можно немного модифицировать: $$A = LDL^T$$$D$ - диагональная матрица, диагональ которой является квадратом диагонали матрицы $L$ в исходном разложении - соответственно, у матрицы $L$ теперь будет единичная диагональ. С таким методом мы избавляемся от вычисления корней диагональных элементов в процессе факторизации. Кроме того, для некоторых неопределенных матриц существует $LDL$-разложение (в отличие от разложения Холецкого). Сложность сохраняется.

\textbf{$QR$-разложение}: $$ A = QR$$ где $Q$ - унитарная матрица ($Q^T Q = QQ^T = E$), $R$ - верхнетреугольная матрица. Разложение существует для любой матрицы (более того, существует даже для прямоугольной размера $m \times n$  ($m > n$) - тогда получаем $Q$ размера $m \times m$ и верхнетреугольную $R$ размера $m \times n$). Разложение чаще всего применяется для решения задачи наименьших квадратов, также используется в одном из алгоритмов для поиска собственных чисел и векторов. Стоимость вычисления $\approx \frac{2}{3}n^3$.

\textbf{Сингулярное разложение}:
$$ M = U \Sigma V^T$$ где $M$ - произвольная матрица размера $m \times n$, $\Sigma$ - диагональная матрица размера $m \times n$, составленная из сингулярных чисел $M$, матрицы $U$ ($m \times m$) и $V$ ($n \times n$) являются унитарными и составлены из левых и правых сингулярных векторов соответственно. Применяется для нахождения псевдообратной матрицы (используется в МНК) или низкоранговых приближений. Сложность - $O(m^2n + mn^2 + n^3)$, в квадратном случае справедлива оценка $4n^3$.


\textbf{Спектральное разложение}:
$$A = V \Lambda V^{-1}$$
где $A$ - диагонализуемая квадратная матрица размера $n \times n$, имеющая $n$ линейно независимых собственных векторов, $V$ - матрица из собственных векторов-столбцов для $A$, $\Lambda$ - диагональная матрица с соответствующими собственными числами на диагонали. Разложение можно использовать для обращения матриц, решения СЛАУ, нахождения определителя и вычисления функций от матриц (к примеру, матричная экспонента). Сложность $\approx 4n^3$.
