\textbf{Невязку} можно определять по-разному, в дальнейшем будут использоваться все виды определений:
\begin{enumerate}
    \item $r_k = \|x_k - x_{opt}\|$
    \item $r_k = |f(x) - f_{opt}|$
    \item $r_k = \|\nabla f(x_k)\|$
\end{enumerate}
Теперь рассмотрим различные виды сходимости. \\
Опреления из лекции:

$\{r_k\}$ сходится \textbf{Q-сублинейно}, если \[\varlimsup\limits_{k\to\infty} \dfrac{r_{k+1}}{r_k} = 1.\]

$\{r_k\}$ сходится \textbf{Q-линейно}, если $\exists c\in(0; 1): $
\[\varlimsup\limits_{k\to\infty} \dfrac{r_{k+1}}{r_k} = c.\]

$\{r_k\}$ сходится \textbf{Q-суперлинейно}, если \[\varlimsup\limits_{k\to\infty} \dfrac{r_{k+1}}{r_k} = 0.\]


$\{r_k\}$ сходится \textbf{Q-квадратично}, если $\exists M>0:$ \[r_{k+1} \leq M r_k^2\ \ \ \forall k \geq k_0\]

$\{r_k\}$ сходится \textbf{R-линейно}, если $\exists$ монотонная последовательность $\{\eta_k\}$: $r_{k+1} \leq \eta_k$ и $\{\eta_k\}$ сходится Q-линейно.
\bigskip

Определения из конспекта:\\
$\{r_k\}$ сходится \textbf{q-линейно} с параметром $q\in (0; 1)$, если существует такое $C > 0$, что

\[r_k \leq Cq^k\ \ \ \forall k \geq m.\]

При этом если существует хотя бы одно такое $q$, то последовательность сходится \textbf{линейно}.

\underline{Пример:} $r_k = \dfrac{1}{3^k}$ и $r_k = \dfrac{4}{3^k}$ сходятся линейно с любым параметром $q \in \left[\dfrac{1}{3}; 1\right)$. При этом константа линейной сходимости (точная нижняя грань множества из всех $q$) для обеих последовательностей равна $\dfrac{1}{3}$.

$\{r_k\}$ сходится \textbf{сублинейно}, если $\{r_k\}$ сходится к 0, но не обладает линейной сходимостью. Если же, наоборот, $\{r_k\}$ обладает линейной сходимостью, и константа линейной сходимости равна 0, говорят, что $\{r_k\}$ сходится \textbf{суперлинейно} или \textbf{сверхлинейно}.

\underline{Пример:} Последовательность $r_k = \dfrac{1}{k}$ не обладает линейной сходимостью, однако сходится к 0, значит, обладает сублинейной сходимостью. \\
Последовательность $r_k = \dfrac{1}{3^{k^2}}$ сходится линейно, при этом параметр сходимости приближается к 0 сколь угодно близко, значит, константа линейной сходимости равна 0, значит, данная последовательность сходится также суперлинейно.

