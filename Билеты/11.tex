\textit{Возможно, один из самых простых билетов} \\

Хотим придумать что-то для функции, которая будет дифференцируемой не в каждой точке. Решение - коль скоро мы не можем подсчитать сам дифференциал, можно подсчитать в точке какой-нибудь его аналог. Спойлер: этот аналог мы будем называть субдифференциалом. \\

\textbf{Definition 1}

Пусть выполнено, что $\forall x \exists s: F(x) \geq F(x_0) + s^T (x - x_0)$

Тогда вектор $s^T$ будем называть субградиентом функции $F$ в точке $x_0$ \\

\textbf{Definition 2}

${s: \forall x \exists s: F(x) \geq F(x_0) + s^T (x - x_0)}$

Множество всех субградиентов в точке $x_0$ - субдифференциал, обозначение $\partial F(x_0)$ \\

\textbf{Утверждение 0}. Если функция дифференцируема в точке $x_0$, то субдифференциал будет состоять из единственного вектор - градиента в этой точке $x_0$

\textbf{Утверждение 1}. Субдифференциал - выпуклое множество

Доказывается по определению субдифференциала. Выписываем два определения субдифференциала и складываем неравенства с коэффициентами $\alpha$ и $(1-\alpha)$

\textbf{Утверждение 2}. Необходимое условие минимума в точке $x_0$ аналогично дифференцируемому случаю, только теперь надо проверять, что $0 \in \partial F(x_0)$. Утверждение 1 позволяет решать задачу необходимого условия это геометрически \\

Сам шаг метода звучит просто: это градиентный метод, только вместо градиента в точке берём какой-нибудь субградиент из субдифференциала (при этом по Утверждению 0 этот "какой-нибудь" субградиент будет в точности совпадать с градиентом).

Проблема: мы теперь не можем использовать правило Армихо/Вульфа, т.к. существуют примеры того, что двигаясь вдоль антисубградиента мы будем увеличивать функцию (в видосике он приводит пример того, что линии уровня из себя представляют сплющенные ромбики, задача задаётся формулой $F(x_1, x_2) = |x_1| + \gamma * |x_2| \to min$).

Альтернативные способ задания такой последовательности:

1) $\alpha_k = \alpha_0$ - если мы возьмём константную длину шага, метод сходиться не будет вообще

2) $\alpha_k = \cfrac{\alpha_0}{||z_k||}$ - делим на норму субградиента. Будем ближе к оптимуму, но сходимости всё ещё нет

3) \{$\alpha_k$\} - выбрать такую последовательность, что $\sum_{i = 0}^{\infty} \alpha_i = \infty$, однако $\sum_{i = 0}^{\infty} \alpha_i^2 < \infty$

Пример: $\alpha_k = \alpha_0 / \sqrt{k + 1}$ или такая последовательность, которая сходится к нулю, но не очень быстро \\

Критерий останова: его нет. Просто идём максимальное количество итераций. В памяти храним лучшую точку, так как метод не гарантирует сходимости к минимуму.

Скорость сходимости - $O(\cfrac{1}{\sqrt{k}})$

Примеры субдифференциального исчисления лучше всего смотреть в семинарской записи, там хорошо написано: \href{https://drive.google.com/file/d/175GWh8qsMg6rVO_xkjDvWbmRG_ALzIbm/view}{Примеры}
